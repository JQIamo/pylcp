%% This template can be used to write a paper for
%% Computer Physics Communications using LaTeX.
%% For authors who want to write a computer program description,
%% an example Program Summary is included that only has to be
%% completed and which will give the correct layout in the
%% preprint and the journal.
%% The `elsarticle' style is used and more information on this style
%% can be found at
%% http://www.elsevier.com/wps/find/authorsview.authors/elsarticle.
%%
%%
%%\documentclass[preprint,12pt]{elsarticle}

%% Use the option review to obtain double line spacing
%% \documentclass[preprint,review,12pt]{elsarticle}

%% Use the options 1p,twocolumn; 3p; 3p,twocolumn; 5p; or 5p,twocolumn
%% for a journal layout:
%% \documentclass[final,1p,times]{elsarticle}
%% \documentclass[final,1p,times,twocolumn]{elsarticle}
%% \documentclass[final,3p,times]{elsarticle}
%% \documentclass[final,3p,times,twocolumn]{elsarticle}
%% \documentclass[final,5p,times]{elsarticle}
\documentclass[final,5p,times,twocolumn]{elsarticle}

\usepackage{graphicx}
\usepackage{amsmath}
\usepackage{amssymb}
\usepackage{braket}

%% natbib.sty is loaded by default. However, natbib options can be
%% provided with \biboptions{...} command. Following options are
%% valid:

%%   round  -  round parentheses are used (default)
%%   square -  square brackets are used   [option]
%%   curly  -  curly braces are used      {option}
%%   angle  -  angle brackets are used    <option>
%%   semicolon  -  multiple citations separated by semi-colon
%%   colon  - same as semicolon, an earlier confusion
%%   comma  -  separated by comma
%%   numbers-  selects numerical citations
%%   super  -  numerical citations as superscripts
%%   sort   -  sorts multiple citations according to order in ref. list
%%   sort&compress   -  like sort, but also compresses numerical citations
%%   compress - compresses without sorting
%%
%% \biboptions{comma,round}

% \biboptions{}

%% This list environment is used for the references in the
%% Program Summary
%%
\newcounter{bla}
\newenvironment{refnummer}{%
\list{[\arabic{bla}]}%
{\usecounter{bla}%
 \setlength{\itemindent}{0pt}%
 \setlength{\topsep}{0pt}%
 \setlength{\itemsep}{0pt}%
 \setlength{\labelsep}{2pt}%
 \setlength{\listparindent}{0pt}%
 \settowidth{\labelwidth}{[9]}%
 \setlength{\leftmargin}{\labelwidth}%
 \addtolength{\leftmargin}{\labelsep}%
 \setlength{\rightmargin}{0pt}}}
 {\endlist}


\journal{Computer Physics Communications}

\begin{document}

\begin{frontmatter}

%% Title, authors and addresses

%% use the tnoteref command within \title for footnotes;
%% use the tnotetext command for the associated footnote;
%% use the fnref command within \author or \address for footnotes;
%% use the fntext command for the associated footnote;
%% use the corref command within \author for corresponding author footnotes;
%% use the cortext command for the associated footnote;
%% use the ead command for the email address,
%% and the form \ead[url] for the home page:
%%
%% \title{Title\tnoteref{label1}}
%% \tnotetext[label1]{}
%% \author{Name\corref{cor1}\fnref{label2}}
%% \ead{email address}
%% \ead[url]{home page}
%% \fntext[label2]{}
%% \cortext[cor1]{}
%% \address{Address\fnref{label3}}
%% \fntext[label3]{}

\title{PyLCP: A package for computing complicated laser cooling physics}

%% use optional labels to link authors explicitly to addresses:
%% \author[label1,label2]{<author name>}
%% \address[label1]{<address>}
%% \address[label2]{<address>}

\author[a]{Stephen Eckel\corref{author}}
\author[a]{Daniel Barker}
\author[a]{Eric Norrgard}

\cortext[author] {Corresponding author.\\\textit{E-mail address:} stephen.eckel@nist.gov}
\address[a]{National Institute of Standards and Technology, Sensor Sciences Division, 100 Bureau Dr., Gaithersburg, MD 20899}

\begin{abstract}
%% Text of abstract
A submitted program is expected to be of benefit to other physicists or physical chemists, or be an exemplar of good programming practice, or illustrate new or novel programming techniques which are of importance to some branch of computational physics or physical chemistry.

Acceptable program descriptions can take different forms. The following Long Write-Up structure is a suggested structure but it is not obligatory. Actual structure will depend on the length of the program, the extent to which the algorithms or software have already been described in literature, and the detail provided in the user manual.

Your manuscript and figure sources should be submitted through the Elsevier Editorial System (EES) by using the online submission tool at \\
http://www.ees.elsevier.com/cpc.

In addition to the manuscript you must supply: the program source code; job control scripts, where applicable; a README file giving the names and a brief description of all the files that make up the package and clear instructions on the installation and execution of the program; sample input and output data for at least one comprehensive test run; and, where appropriate, a user manual. These should be sent, via email as a compressed archive file, to the CPC Program Librarian at cpc@qub.ac.uk.

\end{abstract}

\begin{keyword}
%% keywords here, in the form: keyword \sep keyword
keyword1; keyword2; keyword3; etc.

\end{keyword}

\end{frontmatter}

%%
%% Start line numbering here if you want
%%
% \linenumbers

% Computer program descriptions should contain the following
% PROGRAM SUMMARY.

{\bf PROGRAM SUMMARY}
  %Delete as appropriate.

\begin{small}
\noindent
{\em Program Title: PyLCP}                                          \\
{\em Licensing provisions(please choose one): GPLv3}                                   \\
{\em Programming language: Python}                                   \\

{\em Nature of problem(approx. 50-250 words):}\\
  %Describe the nature of the problem here. \\
{\em Solution method(approx. 50-250 words):}\\
  %Describe the method solution here.
{\em Additional comments including Restrictions and Unusual features (approx. 50-250 words):}\\
  %Provide any additional comments here.
   \\

\begin{thebibliography}{0}
\bibitem{1}Reference 1         % This list should only contain those items referenced in the
\bibitem{2}Reference 2         % Program Summary section.
\bibitem{3}Reference 3         % Type references in text as [1], [2], etc.
                               % This list is different from the bibliography at the end of
                               % the Long Write-Up.
\end{thebibliography}
* Items marked with an asterisk are only required for new versions
of programs previously published in the CPC Program Library.\\
\end{small}


%% main text
\section{Introduction}
\label{sec:intro}
Laser cooling is ubiquitous in modern atomic physics.  The 

Here we introduce a python-based program that computes the movement of atoms or molecules with complex level structures in arbitrary optical (laser) and magnetic fields.  The fundamental piece of the program.

\section{The optical Bloch equations}
Here, we consider the generic problem of coupling $N$ states together in arbitrary optical and magnetic fields.
We group these states into manifolds: a collection of states that are degenerate or nearly degenerate, e.g., the $^2S_0$ states of an alkali atom, the ro-vibrational states of molecules, etc.
We denote the $i$th state and its manifold index $n$ by by $\ket{i, n}$.
The manifolds will be useful for both defining appropriate rotating frames and for applying the rotating wave approximation.

The full Hamiltonian is given by
\begin{equation}
    \label{eq:obe:generic_hamiltonian}
    \hat{H} = \hat{H}_\text{atom} + \hat{H}_\text{field} -
    \hat{\boldsymbol{d}}\cdot\hat{\mathbf{E}} -
    \hat{\boldsymbol{\mu}}\cdot\hat{\mathbf{B}}.
\end{equation}
The field component of the Hamiltonian is given by
\begin{equation}
    \label{eq:obe:field}
    \hat{H}_\text{field} = \int \left(\epsilon_0 \hat{\mathbf{E}} +
    \frac{\hat{\mathbf{B}}}{2}\right)\ dV
\end{equation}
where $\hat{\mathbf{E}}$ is the electric field operator, $\hat{\mathbf{B}}$ is the magnetic field operator.
The atomic Hamiltonian is 
\begin{equation}
    \hat{H}_\text{atom} = \frac{P^2}{2M} + \hat{H}_\text{internal}
\end{equation}
where $\hat{H}_\text{internal}$ describes the internal structure of the atom.
In general, it has the form
\begin{equation}
	\hat{H}_\text{int} = \left[\hbar(\omega_{M,n} + \omega_i)\ket{i,n}\Bra{i, n}\right]
\end{equation}
where $\omega_{M,n}$ is the offset frequency of the $n$th manifold and $\omega_i$ is the $i$th's state frequency relative to $\omega_{M,n}$ and we use the Einstein summation convention.  Manifolds are connected only through $\hat{\mathbf{d}}\cdot\hat{\mathbf{E}}$ component of the Hamiltonian; $\hat{H}_\text{int}$ and $\hat{\boldsymbol{\mu}}$ only act on the subspace of each manifold.

Our goal is to find the evolution of the atom operators $\rho_{ij} = \ket{i}\Bra{j}$.  (We suppress the manifold index when it is not relevant.)  In the Heisenburg picture, the operators $\hat{O}$ evolve as
\begin{equation}
	\label{eq:heisenburg_evolution}
	\frac{\partial \hat{O}}{\partial t}  = \frac{i}{\hbar}[\hat{H}, \hat{O}].
\end{equation}
If the fields were treated classically, this equation would have to amended in order to take into account decays.  Instead, if we quantize the electric fields, derive the equations of motion, and then apply appropriate radiation reaction approximations, we can derive the full optical Bloch equations with decay from Eq.~\ref{eq:heisenburg_evolution}.  The magnetic field $\mathbf{B}$ is assumed to be a classical field; we will not consider quantizing it.  We must pay special focus to the electric field, however, for it both shifts the internal Hamiltonian when transformed into the necessary rotating frame(s) and creates the necessary decay channels.

For the electric field, $\mathbf{E}$ could be comprised of multiple modes.  We group those modes by which modes drives transitions between manifolds $n\rightarrow m$.  Within each manifold, in addition to the mean frequency $\omega_{n\rightarrow m}$, there could then be multiple modes, which we then index by $\omega_p$.  Thus,
\begin{equation}
	\mathbf{E} = \mathbf{E}_{n\rightarrow m, p} e^{-i (\omega_{n\rightarrow m} +\omega_p) t} + \mathbf{E}_{n\rightarrow m, p}^\dagger e^{i (\omega_{n\rightarrow m} +\omega_p) t}
\end{equation}
Here $\mathbf{E}$ represents a destruction operator of the mode $n\rightarrow m, p$.
To determine the decays, we must apply a radiation reaction approximation.  Classicaly, the radiation reaction field is
\begin{equation}
	\mathbf{E}_{RR} = \frac{1}{6\pi\epsilon_0 c^3}\frac{d^3\mathbf{d}}{dt^3}.
\end{equation}
The dipole moment $\mathbf{d}$ will oscillate with all frequency components contained in the drive.  Thus, for each frequency mode, we must take $d$ to have an $e^{-i (\omega_{n\rightarrow m}+\omega_p)t}$ oscillation
\begin{equation}
	\mathbf{E}_{n\rightarrow m,p} = \mathbf{E}_{0,n\rightarrow m,p} + \frac{i (\omega_{n\rightarrow m}+\omega_p)^3}{6\pi\epsilon_0c ^3}e^{-i\delta_{n\rightarrow m, p}t} \mathbf{d}^{nm}_{ij}\ket{i}\Bra{j}
\end{equation}
where the two atomic operators each contribute their preferred rotation, yielding the total oscillation of $\delta_{n\rightarrow m, p} = (\omega_{n\rightarrow m}+\omega_p)-(\omega_{R,m}-\omega_{R,n})$.
We then note that for each manifold, $\omega_{n\rightarrow m}\gg \omega_p$, so we make the substitution
\begin{equation}
	\label{eq:total_e_field_operator}
	\mathbf{E}_{n\rightarrow m,p} = \mathbf{E}_{0,n\rightarrow m,p} + \frac{i k_{n\rightarrow m}^3}{6\pi\epsilon_0 }e^{-i\delta_{n\rightarrow m, p}t} \mathbf{d}^{nm}_{ij}\ket{i}\Bra{j}
\end{equation}
where $k_{n\rightarrow m} = \omega_{n\rightarrow m}/c$.  Clearly the $\rho_{ij}$ operators must commute with all $\mathbf{E}_{n\rightarrow m,p}$, as they are different physical observables.  On the other hand, given that $\rho_{ij}$ does not commute with the second term in (\ref{eq:total_e_field_operator}), it must also not commute with the first.

We similarly expand the dipole operator
\begin{equation}
	\mathbf{d} = \sum_{n,m}\mathbf{d}^{nm}_{ij}\ket{i}\Bra{j} + \mathbf{d}^{*nm}_{ji}\ket{j}\Bra{i}.
\end{equation}
In general, the ${d}^{nm}_{ij}$ are dependent on reduced matrix elements and Clebsch-Gordon coefficients that determine the transitions between manifolds $n$ and $m$.  The dipole elements are grouped by manifold such that $d^{nm}_{ij}$ only operates on states within the manifolds $n$ and $m$, i.e., $d^{nm}_{ij}=0$ if $i\notin m,n$ or  $j\notin m,n$.  We will not consider any specific form of $d^{nm}_{ij}$, but instead focus on deriving the optical Bloch equations for any generic form.

For each manifold, we will assume that all states in that manifold rotate at a preferred frequency $\ket{i,n} \rightarrow e^{i \omega_{R,n} t}\ket{i, n}$.  
We choose the $\omega_{R,n}$ such that their differences $\omega_{R,n}-\omega_{R,m} \approx \omega_{M,n}-\omega_{M,m} \approx \omega_{n\rightarrow m}$ for all combinations of $n$ and $m$.
This choice places each manifold into an appropriate rotating frame.
Under this transformation of the state vectors, the internal Hamiltonian becomes
\begin{equation}
	\hat{H}_\text{internal} = -\hbar \delta^H_{i, n} \ket{i,n}\Bra{i, n},
\end{equation}
where we define $\delta^H_{n, i} = \omega_{R,n}-(\omega_{M, n}+\omega_i)$.
incorporating the shift into the rotating frame into the internal Hamiltonian.  We also define $\delta^L_{n\rightarrow m, p} = (\omega_{n\rightarrow m} + \omega_p) - (\omega_{R,m} - \omega_{R,n})$.  Making the rotating wave approximation (neglecting terms oscillating at optical frequencies), we find, keeping only energy conserving terms,
\begin{eqnarray}
	\mathbf{d}\cdot\mathbf{E} & = & \left(\mathbf{d}^{nm}_{ij}\cdot\mathbf{E}^\dagger_{n\rightarrow m,p} e^{-i\delta^L_{n\rightarrow m, p}  t} \right)  \ket{i}\Bra{j} +  \nonumber \\
	& & ~~~~~~ \ket{j}\Bra{i} \left((\mathbf{d}^{nm}_{ij})^\dagger \cdot \mathbf{E}_{n\rightarrow m,p}e^{i\delta^L_{n\rightarrow m, p}t}\right).\label{eq:d_dot_E}
\end{eqnarray}
Note that even because $\mathbf{E}$ commutes with $\rho_{ij}$, we can place the operators in any order.  We have chosen normal order: the first operator to apply to the wavefunction is the destructor operator of either the atom or the field and the second operator is constructor operator.  This operator ordering is required in order for the radiation reaction approximation to produce the correct decay rate $\Gamma$. 

It is perhaps instructive to consider a couple examples of this construction of the rotating frame(s).  Consider first a standard two level system, with indices $i=g$ and $j=e$ and energies $\omega_g=0$ and $\omega_e$, being driven by a single electric field with total frequency $\omega$.   (Here, we drop the $k$, $n$, and $m$ subscripts.)  Let us take $\omega_{R,g}=0$ and $\omega_{R,e}=\omega_r$.  Then the detuning on the Hamiltonian, $\delta^{H}_e = \omega_R - \omega_e$ and $\delta^L_{g\rightarrow e} = \omega - \omega_R$.  The total detuning of the laser from the excited state is then given by $\delta = \delta^L_{g\rightarrow e}+\delta^H_e = \omega-\omega_e$.  In this way, we can split the detuning between lasers and Hamiltonian in whichever way yields best computational efficiency for the particular problem at hand.

Next, consider a three-level manifold $\Lambda$-system with a single state in each manifold.  Let us label the manifolds as $g$, $r$ and $e$ in order of overall energy, and we drop the unnecessary substate subscripts.  We address this system with two lasers, one tuned closely to $g\rightarrow e$ with frequency $\omega_{g\rightarrow e}$ and the other tuned closely to $r\rightarrow e$ with frequency $\omega_{r\rightarrow e}$.  We now choose $\omega_{r,g} = 0$, and the relevant detunings are then
\begin{eqnarray}
	\delta^H_g & = & 0 \\
	\delta^H_e & = & \omega_{r,e} - \omega_e \\
	\delta^H_r & = & \omega_{r,r} - \omega_r \\
	\delta^L_{g\rightarrow e} & = & \omega_{g\rightarrow e} - \omega_{r,e} \\
	\delta^L_{r\rightarrow e} & = & \omega_{r\rightarrow e} - (\omega_{r,e}-\omega_{r,r})
\end{eqnarray}
By choosing $\omega_{r,e} = \omega_{g\rightarrow e}$ and $\omega_{r,r} = \omega_{g\rightarrow e} - \omega_{r\rightarrow e}$, one recovers the textbook example of the three level system with detunings shifted onto the Hamiltonian.  As with the two level system above, one can split the detunings between lasers and Hamiltonian in whichever way yields best computational efficiency for the particular problem at hand.

Before applying the radiation reaction approximation, we must first find the equations of motion.
Inserting Eq.~\ref{eq:d_dot_E} into Eq.~\ref{eq:heisenburg_evolution}, and using $\rho_{ij}\rho_{kl} = \rho_{il}\delta_{jk}$, where $\delta_{ij}$ is the Kronicker delta function.
\begin{eqnarray}
	\hbar\frac{\partial \rho_{ij}}{\partial t} 
	& = & -i\left(\mathbf{d}^{nm}_{ki}\cdot\mathbf{E}^\dagger_{n\rightarrow m,p} e^{-i\delta^L_{n\rightarrow m, p}  t} \right)  \ket{k}\Bra{j} +  \nonumber \\
	& & -i \ket{k}\Bra{j} \left((\mathbf{d}^{nm}_{ik})^\dagger \cdot \mathbf{E}_{n\rightarrow m,p}e^{i\delta^L_{n\rightarrow m, p}t}\right) \nonumber \\
	& & +i \left(\mathbf{d}^{nm}_{jk}\cdot\mathbf{E}^\dagger_{n\rightarrow m,p} e^{-i\delta^L_{n\rightarrow m, p}  t} \right)\ket{i}\Bra{k}  \nonumber \\
	& & +i \ket{i}\Bra{k} \left((\mathbf{d}^{nm}_{kj})^\dagger \cdot \mathbf{E}_{n\rightarrow m,p}e^{i\delta^L_{n\rightarrow m, p}t}\right) \label{eq:E_coherent_ev}
\end{eqnarray}
Once again, we have maintained normal operator order.

Substituting for each mode $\omega_k$ in the manifold and focusing on the real part of evolution
\begin{eqnarray}
	\hbar\text{Re}\left[\frac{\partial \rho_{ij}}{\partial t}\right] & = &  \frac{k_{n\rightarrow m}^3}{6\pi\epsilon_0}\left[
	-\mathbf{d}_{ki}\cdot \mathbf{d}^\dagger_{lm}\rho_{lm}\rho_{kj} 
	+\rho_{kj}\mathbf{d}_{ki}^\dagger \cdot \mathbf{d}_{lm}\rho_{lm}  \right. \nonumber \label{eq:decays_step_one} \\
	& & \left. + \mathbf{d}_{jk}\cdot \mathbf{d}^\dagger_{lm}\rho_{lm}\rho_{ik}
	- \rho_{ik}\mathbf{d}_{jk}^\dagger \cdot \mathbf{d}_{lm}\rho_{lm}
	\right] \\
	& = & \frac{k_{n\rightarrow m}^3}{6\pi\epsilon_0 }\left[
	-\mathbf{d}_{ki}\cdot \mathbf{d}^\dagger_{lk}\rho_{lj} 
	+ \mathbf{d}_{ki}^\dagger \cdot \mathbf{d}_{jl}\rho_{kl} \right. \nonumber \\
	& & \left. + \mathbf{d}_{jk}\cdot \mathbf{d}^\dagger_{li}\rho_{lk}
	- \mathbf{d}_{jk}^\dagger \cdot \mathbf{d}_{kl}\rho_{il} \right] \label{eq:decay}
\end{eqnarray}
which determines the decays.   {\it Note that there is an index flip in the second and fourth terms between (\ref{eq:decays_step_one}) and (\ref{eq:E_coherent_ev}), but that has to do with $(\mathbf{d}^{n\rightarrow m}_{ij})^\dagger = (\mathbf{d}^*)^{n\rightarrow m}_{ji}$ in the code.  There might be something wrong here: doing the math properly, there is $N$ modes addressing a manifold and so I would nominally expect a factor of $N$ to pop out when I do the sum.}

The coherent evolution (imaginary part) is determined by directly evaluating the commutator. The result is is
\begin{equation}
	\frac{\partial \rho}{\partial t} = -\frac{i}{\hbar}[\rho, H]
\end{equation}
Specifically, for the $\ket{i_n}\Bra{j_m}$ component,
\begin{equation}
	\dot{\rho}^{ij}_{nm} = -\frac{i}{\hbar}[\rho, H]_{i_n, j_m} = -\frac{i}{\hbar}\left[\rho^{ik}_{np}H^{kj}_{pm} - \rho^{ki}_{pm}H^{ik}_{np}\right]
\end{equation}
where Einstein summation is assumed.

The particle's semiclassical motion can be calculated through
\begin{equation}
	\ddot{\mathbf{r}} = -\frac{1}{m} \left\{\nabla H\right\} + \mathbf{a} = \frac{1}{m}\left\{\nabla (\mathbf{d} \cdot \mathbf{E} + \mathbf{\mu}\cdot \mathbf{B})\right\} + \mathbf{a}
\end{equation}
where $\ddot{r}$ is the acceleration of the particle, $\mathbf{a}$ is a constant acceleration, and [...].

Momentum diffusion [...]. Random scattering [...]

\subsection{Representation of the Hamiltonian}
In {\tt pylcp}, we represent this Hamiltonian as a series of blocks, with each block containing a manifold of states (e.g., Zeeman sub-levels or a manifold of hyperfine states). A completely generic basis set vector can then be written as
\begin{equation}
    \ket{\phi} = \left(\begin{array}{c} \begin{array}{c} \ket{l_1} \\ \vdots \\ \ket{l_{l_{N_l}}} \end{array} \\ \begin{array}{c} \ket{n_1} \\ \vdots \\ \ket{n_{N_n}} \end{array} \end{array}\right)\ ,
\end{equation}
where $\ket{l_i}$ are the eigenstates of the first (ground) manifold and $\ket{n_k}$ are
the eigenstates of the most excited manifold.  With this basis vector, the term $\boldsymbol{\mu}_l\cdot\mathbf{B}$ is the field dependent term that mixes states within a given manifold $l$ and $\vec{d}_{lm}\cdot\mathbf{E}_{lm}$ is the field dependent term that couples states of different manifolds.  Using labels $i=g,e$ for the two extreme manifolds, the Hamiltonian blocks look like
\begin{equation}
    \label{eq:ham_matrix_form}
    H_\text{atom} = \left(
    \begin{array}{ccc}
    (H_g - \boldsymbol{\mu}_g\cdot \mathbf{B}) & \cdots & (\boldsymbol{d}_{ge}\cdot\mathbf{E}_{ge}) \\
    \vdots & \ddots & \vdots  \\
    (\boldsymbol{d}_{ge}^\dagger\cdot\mathbf{E}_{ge}^*) & \cdots & (H_e+\boldsymbol{\mu}_e\cdot \mathbf{B})
    \end{array}\right),
\end{equation}
where each element in the matrix $H_g$, $H_e$, etc. is itself a matrix.  In general, the the electric fields driving transitions between manifolds are distinct, which is why the electric field gains a specific label in Eq.~\ref{eq:ham_matrix_form}.

To specify the problem, the user defines a Hamiltonian like Eq. (5) by providing the requisite $H_0$, $\boldsymbol{\mu}_l$, and $\boldsymbol{d}_{lm}$ and combining them in the {\tt hamiltonian} class.  The class creates and stores the block structure of the Hamiltonian, and also contains methods that are useful for its manipulation.  The vectors $\boldsymbol{\mu}$ and $\boldsymbol{d}$ are represented in spherical polar coordinates, allowing for translation into $\sigma^{\pm}$ circular polarization and $\pi$ polarizations.  For these vectors, we denote the component $q$ as $d^q$.

\subsection{Fields}
In its construction, {\tt pylcp} assumes a that the Hamiltonian is written in a co-rotating frame with a frequency close to that of the laser light, with the number of rotating frames equal to $N_m-1$.  The electric field has the form
\begin{equation}
	\mathbf{E}_{lm} = \frac{1}{2}\hat{\boldsymbol{\epsilon}}(r, t) E_{lm}(r, t)e^{i\mathbf{k}(r,t)\cdot\mathbf{r}-i \Delta t + \phi(t)}
\end{equation}
where the complex conjugate term is neglected by the rotating wave approximation.  
The user specifies all components of the laser field: the polarization vector $\hat{\boldsymbol{\epsilon}}(r, t)$, the amplitude $E_{lm}(r, t)$, the $\mathbf{k}$ vector, the average detuning between the rotating frame and optical frequency $\Delta$,  and any potential phase modulation of the laser beam $\phi(t)$ in the {\tt laserBeam} class.

Laser beams are collected by which manifold transition they address.

Finally, the magnetic field $\mathbf{B}$ is specified simply as a user-defined function  of space and time.

\subsection{Units}

For {\tt pylcp}, $\hbar=1$ so that angular frequencies and energies are equivalent. Thus, the user specifies $H_0/\hbar$, rather than $H_0$.

For other units, we will specifically separate out the units, and dimensionless number shall be denoted with a bar.
For example, a position $x$ is related to its dimensionless counterpart through $x = \bar{x}x_0$, where $x_0$ is the unit of $x$.
When forming a complete Hamiltonian using the hamiltonian class, the user has the ability to set the units by specifying base units of length $x_0$, time $t_0$, magnetic field $B_0$ and mass $m$.  

To understand the units associated with $d\cdot E$, let us briefly consider the two level system.  The excited state decay rate is given by $\Gamma = k^3(d d^*+d^* d)/6 \pi\epsilon_0\hbar$.  Thus, we define a normalized $d=\bar{d} d_0$, where $d_0 = \sqrt{3 \pi \epsilon_0 \hbar \Gamma/k^3}$.  With this definition, $\bar{d}^*\bar{d}+\bar{d}\bar{d}^* = \Gamma/2$.  For the electric field, we use the on-resonant, two-level saturation parameter $s = I/I_\text{sat} = 2|d_0 E|^2/\hbar^2|\gamma|^2 = 8 d_0 |E|^2/\hbar^2\Gamma^2$, where $\gamma = \Gamma/2$ to define the natural units for $E$.  Inverting, we find that $E = \hbar \Gamma \sqrt{s}/(\sqrt{8} d_0)$.  Thus, $d\cdot E/\hbar = (\bar{d} \Gamma/4) \sqrt{2 s}$.  We note that this definition of the saturation parameter is consistent with the more common $I/I_\text{sat} = 2(\Omega^2/\Gamma^2)$, where, because the Hamiltonian is defined without a factor of 1/2 in the $\mathbf{d}\cdot \mathbf{E}$ term, $\Omega/2=dE$.

For the force, this leaves one additional unit to specify, the mass.  If the
user specifies the length, time, and magnetic field, then
\begin{equation}
    \frac{x_0}{t_0^2} \ddot{\bar{\mathbf{r}}} = \frac{\hbar}{x_0 t_0 m}\left\{ -\bar{\nabla} \bar{H}\right\} + \frac{x_0}{t_0^2}\bar{\mathbf{a}}.
\end{equation}
The default unit selection for a two-manifold system is $t_0=1/\Gamma$ and $x_0=1/k$, which yields for the prefactor $\hbar/x_0 t_0 m = \hbar k \Gamma/m$.   Further taking into account the units on the left hand side we define
\begin{equation}
	\ddot{\mathbf{r}}' = \frac{\hbar t_0}{x_0^2 m}\left\{ -\bar{\nabla} \bar{H}\right\} + \bar{\mathbf{a}}
\end{equation}
which defines the `dimensionless' mass as $\bar{m} = x_0^2 m/\hbar t_0$.

For random scattering, the recoil velocity $\hbar \mathbf{k}/m$ vector must also be specified.  In terms of dimensionless units,
\begin{equation}
	\bar{v}_R \frac{x_0}{t_0} = \frac{\hbar\bar{k}}{m x_0} \rightarrow \bar{v}_R = \frac{\bar{k}}{\bar{m}}\ .
\end{equation}
{\it Once again, I hate ending a paragraph or even a section with a sentence.}

\subsection{Additional details}
The optical Bloch equations are symmetric under population exchange, i.e., $\rho_{l_i, m_j} = \rho^*_{m_j, l_i}$.  The package {\tt pylcp} takes advantage of this symmetry to transform the optical Bloch equations into real and imaginary parts, increasing computational speed.

\section{The Rate Equations}
In order to do a rate equation model, we first consider that we have a set of
eigenstates and that the matrix~\ref{eq:obe:ham_without_fields} is diagonal.  We
follow the construction of the rate equations from Ref.~\cite{Tarbutt2015} and,
to a lesser extent, the Ref.~\cite{Tollett1995}.
In the rate equation model, we consider three things.  The first is the
excitation rate caused by laser $l$
\begin{equation}
    \label{eq:lcp:scattering_rate}
    R^l_{ij} = \Gamma \left(\frac{\Omega^l_{ij}}{\Gamma}\right)^2\frac{1}{1 + 4(\Delta - k\cdot v)^2/\Gamma^2},
\end{equation}
and the decay rate out of the excited state $\Gamma$, and the branching ratio,
\begin{equation}
    \Gamma_{e_j,g_i} = \Gamma \frac{\left|d^{(q)}_{ij}\right|^2}{\left(\sum_i \left|d^{(q)}_{ij}\right|^2\right)}.
\end{equation}
In general, we want to solve for $\dot{N}_{g/e,i}$.  We will do this in units of
$\Gamma$.  For the ground states,
\begin{equation}
    \dot{N}_i = -\sum_{jl} R^l_{ij} (N_i - N_j) + \sum \Gamma_{ij}N_j
\end{equation}
and for the excited states,
\begin{equation}
    \dot{N}_j = - \Gamma N_j + \sum_{jl} R^l_{ij} (N_i - N_j)
\end{equation}
We represent the rate equations as a matrix $R$ such that
\begin{equation}
    \dot{N} = R N
\end{equation}
where $N = (N_{g_1},\dots, N_{g_n}, N_{e_1}, \dots, N_{g_n})$.

\section{Additional details}

\subsection{A word about polarization}
We have thusfar considered the polarization to be contained $E^l_q$.  However,
computing this quantity is not necessarily trivial.  First, one takes the
magnitude out realizing that is related to the intensity, i.e., $E^l_q =
|E|^l\hat{\epsilon}^l_q$.  Second, $\hat{\epsilon}^l_q$ is defined in the
rotational basis of the quantization axis, whereas for an actual laser beam,
the polarization is defined via some other means.

In general, the polarization is a complex-three space vector, which therefore
has 5 independent parameters.  However, because light is a transverse wave and
$\hat{k}$ is defined (with two parameters), that reduces the number of
parameters to three.  We often think about the Poincare sphere or polarization
ellipse at this stage which is defined by only two parameters, but there is a
hidden parameter which is the orientation of the coordinate system in which
the sphere or ellipse is defined.

For the purposes of computation of $\hat{\epsilon}_q\cdot\hat{\epsilon}_l$,
there seems to be multiple ways to define the coordinate system.  Consider, for
example, defining the polarization in terms of right $\epsilon^+$ and left
$\epsilon^-$ polarized light.  Then the most general polarization vector is
$\epsilon_l = a \epsilon^+_l + b \epsilon^-_l$, where $a$ and $b$ are complex
constants and $|a|^2+|b|^2=1$.  There are thus three independent parameters that
specify this polarization.  Note that the phase between $a$ and $b$ determines
the angle in the plane that a linear projection will point and that
$\epsilon^+_l=\sigma^+$ and $\epsilon^-_l=\sigma^-$ if the magnetic field is
parallel to $\hat{k}$.

If $\hat{k}$ is not parallel to the quantization axis, then we need to rotate
one coordinate system (the laser and polarization) into the other (where the
quantization is done).  Imagine that the Euler angles doing the wrapping are
$\alpha$-$\beta$-$\gamma$ in the z-y-z reference.  In this case, the vector is
rotated through a Wigner rotation matrix, whose elements are given by
\begin{equation}
  \mathcal{D}^j_{m'm}(\alpha,\beta,\gamma)
    = \Braket{jm'|\mathcal{R}(\alpha,\beta,\gamma)|jm}
    = e^{i m'\alpha} d^j_{m' m}(\beta)e^{-im\gamma}
\end{equation}
where $d^j_{m'm}$ is the small Wigner rotation matrix.  At first glance, the
diagonal nature of matrix for $\alpha$ and $\gamma$ seems to imply that those
angle do not play an important role; all of the relevant physics is simply
related to $\beta$.  Alternatively, this would imply that the phase angle
between the $\sigma^+$ and $\sigma^-$ components of the unrotated light basis.
We can explicitly construct the matrix element, using $d^j_{m'm}$ from Wikipedia
and the relationship $d^j_{m'm} = (-1)^{m-m'}d^j_{mm'} = d^j_{m,m'}$.
\begin{equation}
  \mathcal{D} = \left(\begin{array}{ccc}
  \frac{1+\cos\beta}{2}e^{-i\alpha+i\gamma} & \frac{\sin\beta}{\sqrt{2}}e^{-i\alpha} & \frac{1-\cos\beta}{2}e^{-i\alpha-i\gamma} \\
  -\frac{\sin\beta}{\sqrt{2}}e^{i\gamma} & \cos\beta & -\frac{\sin\beta}{\sqrt{2}}e^{-i\gamma} \\
  \frac{1-\cos\beta}{2}e^{i\alpha+i\gamma} & \frac{\sin\beta}{\sqrt{2}} & \frac{1+\cos\beta}{2}e^{i\alpha-i\gamma}
  \end{array}\right)
\end{equation}
This clearly tells a more complicated story.  The simplest observation might  be
that one of the Euler angles does not matter, at least for the rate equation
model.  In that case, we will square each component and the common phase $e^{\pm
i \alpha}$ will drop out.  This makes sense, as the rotation into the quantization
axis should only have two angles, the third being degenerate because it is an
axis not a full 3-D coordinate system.

Consider the final $\pi$ ($m'=0$) component.  An input beam with equal amounts
of $\sigma^+$ and $\sigma^-$  defines linear polarization, and the relative
phase between the two components defines where that points relative to some
other axis, defined by the overall  phase.  If this differential phase is
$\chi$, then the $\pi$ component would be given by
\begin{equation}
  \hat{\epsilon}_l\cdot\hat{\epsilon_0} = -\frac{\sin\beta}{\sqrt{2}}
  \cos(\gamma+\chi).
\end{equation}
This dependence on the phase angle makes intuitive sense.  Consider a
linearly-polarized wave moving in $+\hat{x}$ and magnetic field along $\hat{z}$.
If the phase angle was such that the two circular components added to produce
linear polarization along $\hat{z}$, the polarization would be completely $\pi$.
On the other hand, if the phase angle was such that the linear polarization was
along $\hat{y}$, then in the spherical basis we would have equal amounts of
$\sigma^+$ and $\sigma^-$ light (with a phase angle in between).

None of this gets us closer to the final result, but it is interesting.

\section{Examples}
Optical pumping.


%% The Appendices part is started with the command \appendix;
%% appendix sections are then done as normal sections
%% \appendix

%% \section{}
%% \label{}

%% References
%%
%% Following citation commands can be used in the body text:
%% Usage of \cite is as follows:
%%   \cite{key}         ==>>  [#]
%%   \cite[chap. 2]{key} ==>> [#, chap. 2]
%%

%% References with bibTeX database:

\bibliographystyle{elsarticle-num}
\bibliography{<your-bib-database>}

%% Authors are advised to submit their bibtex database files. They are
%% requested to list a bibtex style file in the manuscript if they do
%% not want to use elsarticle-num.bst.

%% References without bibTeX database:

% \begin{thebibliography}{00}

%% \bibitem must have the following form:
%%   \bibitem{key}...
%%

% \bibitem{}

% \end{thebibliography}


\end{document}

%%
%% End of file
